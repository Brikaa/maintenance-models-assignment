\documentclass[11pt,a4paper]{article}
\usepackage{hyperref}

\begin{document}

\title{Maintenance models and the quick fix model}
\author{Omar Adel Brikaa - 20206043\thanks{Omar Adel Abdel Hamid Ahmed Brikaa - S5 - brikaaomar@gmail.com}}
\date{}
\maketitle

\tableofcontents

\section{Introduction}
This article briefly discusses what maintenance models are, why we study them,
and what the quick fix model is and suggests advice to abide by when using the quick fix model.
The article assumes the reader already knows what software maintenance means.

\section{What are software maintenance models?}
Software maintenance models are abstract descriptions of:
\begin{itemize}
    \item the sequence of activities in the software maintenance process,
    \item the inputs to each activity,
    \item the outputs produced from each activity,
    \item (optionally) metrics that can be applied on the outputs of each activity to evaluate the maintenance process,
        and
    \item (optionally) functions that can be applied while executing the activities to control the maintenance process.
\end{itemize}

\section{Why do we study software maintenance models?}
The reason behind studying software maintenance can be looked at from two points of views:
\begin{itemize}
    \item the scientific point of view: to describe the maintenance process, and
    \item the engineering point of view: to prescribe enhancements to the maintenance process.
\end{itemize}
Both of these points of views fall under the general point of view: to write better software.

\subsection{The scientific point of view: to describe}
This point of view looks at the software maintenance process in general.
It tries to produce conceptual models of the maintenance process and to research the produced models
so as to understand them, extend them and suggest better alternatives to them.

\subsection{The engineering point of view: to prescribe}
This point of view looks at the software maintenance process in a certain organization.
An organization studies well-known maintenance models that have proven to be successful in other organizations.
It then tries to incorporate such models into its workflow.
Moreover, studying different models also helps the organization decide on which one to use in a new project based on
the properties of the model and the project.

\subsection{The general point of view: to write better software}
The ultimate goal behind the scientific and the engineering point of view is to
reduce the cost of software maintenance in the long-run,
and to increase the speed and ease of future maintenance leading to more efficient maintenance and better software.

\section{What is the quick fix model?}
It is a software maintenance model in which the sequence of activities, their inputs and their outputs are as follows:
\begin{itemize}
    \item Modify the code as soon as a problem is discovered or a change is needed
    \subitem Input: old source code
    \subitem Output: source code after implementing the change
    \item Modify of the accompanying documentation to correspond to the code change
    \subitem Input: old documentation
    \subitem Output: updated documentation that is consistent with the new source code
\end{itemize}
It is used in organizations facing tight deadlines and lack of resources
in order to avoid the cost and time of a full maintenance lifecycle.

\section{Advice when using the quick fix model}
The following pieces of advice do not try to change the core principles of the quick fix model.
They suggest actions to do in order to mitigate some of its problems.

\subsection{Advice to prevent code degradation}
One of the problems of the quick fix model is that no attention is paid to the overall structure of the code.
This leads to code degradation after each iteration in the quick fix model.
One piece of advice to mitigate this problem is to
allocate a known time of the month or the week to restructure or refactor the code.
It could be a time when the organization has free resources and few deadlines.

Another piece of advice is to get developers who thoroughly understand
the overall structure of the code to review the code changes before merging them.
This prevents changes that are inconsistent with the overall code structure from getting into the codebase

\subsection{Advice to keep the documentation updated}
Another problem of the quick fix model is that it is easy to skip modifying the documentation
after doing the code changes.
This leads to outdated documentation that is misleading to new team members.
Similar to the previous problem, one possible mitigation is to
allocate a know time of the month or the week to update the documentation.

Another mitigation is to give a certain developer, or group of developers, the responsibility of updating the
documentation rather than making documentation updating a weak rule that developers may or may not follow.

Moreover, it helps to use automated tools that generate documentation from source code and comments.

\end{document}
